% Chapter Template

\chapter{Conclusiones} % Main chapter title

\label{Chapter5} % Change X to a consecutive number; for referencing this chapter elsewhere, use \ref{ChapterX}

En este capítulo se muestran las conclusiones sobre el trabajo realizado y se presentan algunas mejoras como posible trabajo futuro.

%----------------------------------------------------------------------------------------

%----------------------------------------------------------------------------------------
%	SECTION 1
%----------------------------------------------------------------------------------------

\section{Resultados obtenidos}

En este trabajo se completó el diseño, desarrollo y \textit{testing} de un sistema de gestión de cultivos aeropónicos.

Para evaluar los resultados obtenidos del trabajo es importante destacar los siguientes factores:
\begin{itemize}
	\item Se cumplió con la planificación en tiempo y forma aunque no se siguió exactamente con el orden de desarrollo planteado para cada tarea.
	\item No se manifestó ninguno de los riesgos previamente identificados en la planificación.
	\item Luego de realizar un análisis de los requerimientos, se concluye que todos fueron cumplidos. Sin embargo, como se menciona en la sección \ref{sec:observaciones} se realizaron dos modificaciones en el trabajo que afectaron algunos requerimientos: se decidió reemplazar la aplicación SSR por una PWA y se realizaron modificaciones en los datos a almacenar en el DaaS.
\end{itemize}

Fueron de gran utilidad los conocimientos adquiridos a lo largo de la especialización. A continuación, se enumeran las asignaturas que tuvieron mayor relevancia:
\begin{itemize}
	\item Gestión de proyectos.
	\item Arquitectura de protocolos.
	\item Gestión de grandes volúmenes de datos.
	\item Desarrollo de aplicaciones multiplataforma.
	\item Ciberseguridad en Internet de las Cosas.
\end{itemize}

%\section{Conclusiones generales }

%La idea de esta sección es resaltar cuáles son los principales aportes del trabajo realizado y cómo se podría continuar. Debe ser especialmente breve y concisa. Es buena idea usar un listado para enumerar los logros obtenidos.

%Algunas preguntas que pueden servir para completar este capítulo:

%\begin{itemize}
%\item ¿Cuál es el grado de cumplimiento de los requerimientos?
%\item ¿Cuán fielmente se puedo seguir la planificación original (cronograma incluido)?
%\item ¿Se manifestó algunos de los riesgos identificados en la planificación? ¿Fue efectivo el plan de mitigación? ¿Se debió %aplicar alguna otra acción no contemplada previamente?
%\item Si se debieron hacer modificaciones a lo planificado ¿Cuáles fueron las causas y los efectos?
%\item ¿Qué técnicas resultaron útiles para el desarrollo del proyecto y cuáles no tanto?
%\end{itemize}


%----------------------------------------------------------------------------------------
%	SECTION 2
%----------------------------------------------------------------------------------------

\section{Trabajo futuro}

A continuación se detallan las principales líneas de acción para dar continuidad a este trabajo:

\begin{itemize}
\item Mejorar el \emph{responsive} de las tablas del \emph{frontend}.
\item Incorporar SSR a la PWA.
\item Incorporar sensor de pH de la solución nutritiva.
\item Incorporar sensor de TDS de la solución nutritiva.
\item Incorporar estadísticas y más gráficos para el \emph{dashboard} de las zonas de cultivo.
\item Permitir crear más de un dispositivo por zona de cultivo.
\item Permitir seleccionar qué sensores se quiere asociar a cada dispositivo de las zonas de cultivo.
\item Desarrollar pruebas unitarias para el \emph{broker}.
\item Añadir multidioma al sistema utilizando i18n \citep{WEBSITE:ANGULARI18N}.
\item Permitir seleccionar qué unidades se quiere utilizar en las mediciones, por ejemplo: Fahrenheit o Celsius para las temperaturas.
\item Calibrar los sensores del sistema.
\item Realizar pruebas de campo en un entorno real.
\item Integrar el sistema en un gabinete.
\item Desplegar el sistema a un entorno \emph{cloud}.
\end{itemize}